% Este documento destina-se a servir como modelo para a produção de documentos
% de pesquisa do PPGINF/UFPR, como projetos, dissertações e teses. A classe de
% documento se chama "ppginf" (arquivo ppginf.cls) e define o formato básico do
% documento. O texto está organizado em capítulos que são colocados em
% subdiretórios separados. São definidos exemplos para a inclusão de figuras,
% códigos-fonte e a definição de tabelas.
%
% Produzido por Carlos Maziero (maziero@inf.ufpr.br) em Outubro de 2015.
% Adaptado de um modelo anterior construído pelo autor para o PPGIA/PUCPR.

% Opções da classe ppginf:
%
% - defesa  : versão para entregar à banca; tem espaçamento 1,5
%             e omite algumas páginas iniciais (agradecimentos, etc)
% - final   : versão pós-defesa, para enviar à biblioteca;
%             tem espaçamento simples e todas as páginas iniciais.
% - oneside : para impressão somente frente; use quando for gerar
%             somente o PDF, sem impressão.
% - twoside : para impressão frente/verso; use quando for gerar
%             uma versão impressa, para economizar papel.
% - ... (demais opções aceitas pela classe "book")

% Opções default: defesa, oneside
\documentclass[defesa,oneside]{ppginf}	% versão para a defesa
%\documentclass[final,oneside]{ppginf}	% versão final, só em PDF
%\documentclass[final,twoside]{ppginf}	% versão final, em PDF + impresso

% configurações de diversos pacotes, inclusive o fonte principal do texto
\include{packages}

%=====================================================

\begin {document}

% Principais dados, usados para gerar as páginas iniciais.
% Campos não utilizados podem ser removidos ou comentados.

% título :-)
\title{TCC sobre teoria espectral dos grafos}

% palavras-chave e keywords
\pchave{palavra-chave 1, palavra-chave 2, palavra-chave 3}
\keyword{keyword 1, keyword 2, keyword 3}

% autoria
\author{Richard Fernando Heise Ferreira}
\advisor{André Luiz Pires Guedes}
\coadvisor{}
\instit{UFPR}{Universidade Federal do Paraná}

% área de concentração (default do PPGInf, não mudar)
\field{Ciência da Computação}

% local e data
\date{2024}
\local{Curitiba PR}

% imagem de fundo da capa (comentar se não desejar)
\coverimage{0-iniciais/fundo-capa.jpg}

%% Descrição do documento (obviamente, descomentar somente UMA!)

% tese de doutorado
\descr{Tese apresentada como requisito parcial à obtenção do grau de Doutor em Informática, no Programa de Pós-Graduação em Informática, setor de Ciências Exatas, da Universidade Federal do Paraná}

% exame de qualificação de doutorado
%\descr{Documento apresentado como requisito parcial para o exame de qualificação de Doutorado, no Programa de Pós-Graduação em Informática, setor de Ciências Exatas, da Universidade Federal do Paraná}

% dissertação de mestrado
%\descr{Dissertação apresentada como requisito parcial à obtenção do grau de Mestre em Informática, no Programa de Pós-Graduação em Informática, setor de Ciências Exatas, da Universidade Federal do Paraná}

% exame de qualificação de mestrado
%\descr{Documento apresentado como requisito parcial para o exame de qualificação de Mestrado, no Programa de Pós-Graduação em Informática, setor de Ciências Exatas, da Universidade Federal do Paraná}

% trabalho de conclusão de curso
\descr{Trabalho apresentado como requisito parcial à conclusão do Curso de Bacharelado em Ciência da Computação pela Universidade Federal do Paraná, setor de Ciências Exatas}

% trabalho de disciplina
%\descr{Trabalho apresentado como requisito parcial à conclusão da disciplina XYZ no Curso de Bacharelado em XYZ, setor de Ciências Exatas, da Universidade Federal do Paraná}

%=====================================================

% define estilo das páginas iniciais (capas, resumo, sumário, etc)
\frontmatter
\pagestyle{frontmatter}

% define capa e folha de rosto
\titlepage

% páginas que só aparecem na versão final (a inclusão é automática)
% - IMPORTANTE - IMPORTANTE - IMPORTANTE - IMPORTANTE -
%
% O conteúdo exato da ficha catalográfica é preparada pela Biblioteca Central
% da UFPR, a pedido da secretaria do PPGINF. Não "invente" um conteúdo para ela,
% se informe a respeito com a secretaria do programa.

\begin{ficha}	% só gera conteúdo se for na versão final

% inclusão da ficha catalográfica final (arquivo PDF)
\includepdf[noautoscale]{0-iniciais/catalografica.pdf}

\end{ficha}

%=====================================================
	% ficha catalográfica
\include{0-iniciais/aprovacao}		% folha de aprovação
\include{0-iniciais/dedica}		% dedicatória
\include{0-iniciais/agradece}		% agradecimentos

% resumo (português) e abstract (inglês), nesta ordem
\begin{resumo}

Teoria Espectral dos Grafos é um tema que permeia a álgebra linear, química quântica, computação e matemática -- além de muitas outas áreas que tomam suporte nestas. Este trabalho busca explorar essa teoria de modo a expor suas capacidades e testar sua aplicabilidade em grafos reais. 

\end{resumo}


\include{0-iniciais/abstract}

% listas  de figuras, tabelas, abreviações/siglas, símbolos
\listoffigures
\listoftables
%=====================================================

% lista de acrônimos (siglas e abreviações)

\begin{listaacron}

\begin{longtable}{p{0.2\linewidth}p{0.7\linewidth}}
DINF & Departamento de Informática\\
Bacharelado & Graduação em Ciência da Computação\\
UFPR & Universidade Federal do Paraná\\
\end{longtable}

\end{listaacron}

%=====================================================
		% ainda deve ser preenchida à mão
\include{0-iniciais/simbolos}		% idem

% sumário
\tableofcontents

%=====================================================

% define estilo do corpo do documento (capítulos e apêndices)
\mainmatter
\pagestyle{mainmatter}

% inclusao de cada capítulo, alterar a gosto (do professor de Metodologia)
\chapter{Exemplo de anexo}

%=====================================================

Os apêndices são uma extensão do texto, destacados deste para evitar descontinuidade na sequência lógica ou alongamento excessivo de determinado assunto ou tópico secundário dentro dos capítulos da dissertação ou da tese. São contribuições que servem para esclarecer, complementar, provar ou confirmar as ideias apresentadas no texto dos capítulos e que são importantes para a compreensão dos mesmos.

Todos os apêndices devem vir após as referências bibliográficas e devem ser enumerados por letras maiúsculas (A, B, C, ...).

%=====================================================

\section{Uma Seção}

\lipsum[20-23]

%=====================================================

\subsection{Uma Subseção}

\lipsum[30-33]

%=====================================================

\subsection{Outra Subseção}

Exemplo de lista simples com dois níveis: Exemplo de lista simples com dois níveis: Exemplo de lista simples com dois níveis: Exemplo de lista simples com dois níveis: Exemplo de lista simples com dois níveis: Exemplo de lista simples com dois níveis: Exemplo de lista simples com dois níveis: Exemplo de lista simples com dois níveis: Exemplo de lista simples com dois níveis.

\begin{itemize}

\item Banana, Banana, Banana, Banana, Banana, Banana, Banana, Banana, Banana, Banana, Banana, Banana, Banana, Banana, Banana, Banana, Banana, Banana, Banana, Banana, Banana, Banana, Banana, Banana.

\begin{itemize}

\item Caturra, Caturra, Caturra, Caturra, Caturra, Caturra, Caturra, Caturra, Caturra, Caturra, Caturra, Caturra, Caturra, Caturra, Caturra, Caturra, Caturra, Caturra, Caturra.

\item da Terra, da Terra, da Terra, da Terra, da Terra, da Terra, da Terra, da Terra, da Terra, da Terra, da Terra, da Terra, da Terra, da Terra, da Terra, da Terra, da Terra, da Terra.

\end{itemize}

\item Laranja, Laranja, Laranja, Laranja, Laranja, Laranja, Laranja, Laranja, Laranja, Laranja, Laranja, Laranja, Laranja, Laranja, Laranja, Laranja.

\begin{itemize}

\item Bahia, Bahia, Bahia, Bahia, Bahia, Bahia, Bahia, Bahia, Bahia, Bahia, Bahia, Bahia, Bahia, Bahia, Bahia, Bahia, Bahia, Bahia.

\item Lima, Lima, Lima, Lima, Lima, Lima, Lima, Lima, Lima, Lima, Lima, Lima, Lima, Lima, Lima, Lima, Lima, Lima, Lima, Lima, Lima, Lima, Lima, Lima, Lima, Lima, Lima, Lima, Lima.

\end{itemize}

\end{itemize}

Exemplo de lista numerada com dois níveis: Exemplo de lista numerada com dois níveis: Exemplo de lista numerada com dois níveis: Exemplo de lista numerada com dois níveis: Exemplo de lista numerada com dois níveis: Exemplo de lista numerada com dois níveis: Exemplo de lista numerada com dois níveis: Exemplo de lista numerada com dois níveis.

\begin{enumerate}

\item Banana, Banana, Banana, Banana, Banana, Banana, Banana, Banana, Banana, Banana, Banana, Banana, Banana, Banana, Banana, Banana, Banana, Banana, Banana, Banana, Banana, Banana, Banana, Banana.

\begin{enumerate}

\item Caturra, Caturra, Caturra, Caturra, Caturra, Caturra, Caturra, Caturra, Caturra, Caturra, Caturra, Caturra, Caturra, Caturra, Caturra, Caturra, Caturra, Caturra, Caturra.

\item da Terra, da Terra, da Terra, da Terra, da Terra, da Terra, da Terra, da Terra, da Terra, da Terra, da Terra, da Terra, da Terra, da Terra, da Terra, da Terra, da Terra, da Terra.

\end{enumerate}

\item Laranja, Laranja, Laranja, Laranja, Laranja, Laranja, Laranja, Laranja, Laranja, Laranja, Laranja, Laranja, Laranja, Laranja, Laranja, Laranja.

\begin{enumerate}

\item Bahia, Bahia, Bahia, Bahia, Bahia, Bahia, Bahia, Bahia, Bahia, Bahia, Bahia, Bahia, Bahia, Bahia, Bahia, Bahia, Bahia, Bahia.

\item Lima, Lima, Lima, Lima, Lima, Lima, Lima, Lima, Lima, Lima, Lima, Lima, Lima, Lima, Lima, Lima, Lima, Lima, Lima, Lima, Lima, Lima, Lima, Lima, Lima, Lima, Lima, Lima, Lima.

\end{enumerate}

\end{enumerate}

Exemplo de lista descritiva com dois níveis: Exemplo de lista descritiva com dois níveis: Exemplo de lista descritiva com dois níveis: Exemplo de lista descritiva com dois níveis: Exemplo de lista descritiva com dois níveis: Exemplo de lista descritiva com dois níveis: Exemplo de lista descritiva com dois níveis: Exemplo de lista descritiva com dois níveis: Exemplo de lista descritiva com dois níveis.

\begin{description}

\item [Banana]: Banana, Banana, Banana, Banana, Banana, Banana, Banana, Banana, Banana, Banana, Banana, Banana, Banana, Banana, Banana, Banana, Banana, Banana, Banana, Banana, Banana, Banana, Banana.

\begin{description}

\item [Caturra]: Caturra, Caturra, Caturra, Caturra, Caturra, Caturra, Caturra, Caturra, Caturra, Caturra, Caturra, Caturra, Caturra, Caturra, Caturra, Caturra, Caturra, Caturra.

\item [da Terra]: da Terra, da Terra, da Terra, da Terra, da Terra, da Terra, da Terra, da Terra, da Terra, da Terra, da Terra, da Terra, da Terra, da Terra, da Terra, da Terra, da Terra.

\end{description}

\item [Laranja]: Laranja, Laranja, Laranja, Laranja, Laranja, Laranja, Laranja, Laranja, Laranja, Laranja, Laranja, Laranja, Laranja, Laranja, Laranja.

\begin{description}

\item [Bahia]: Bahia, Bahia, Bahia, Bahia, Bahia, Bahia, Bahia, Bahia, Bahia, Bahia, Bahia, Bahia, Bahia, Bahia, Bahia, Bahia, Bahia.

\item [Lima]: Lima, Lima, Lima, Lima, Lima, Lima, Lima, Lima, Lima, Lima, Lima, Lima, Lima, Lima, Lima, Lima, Lima, Lima, Lima, Lima, Lima, Lima, Lima, Lima, Lima, Lima, Lima, Lima.

\end{description}

\end{description}

%=====================================================
			% introdução
\chapter{Exemplo de anexo}

%=====================================================

Os apêndices são uma extensão do texto, destacados deste para evitar descontinuidade na sequência lógica ou alongamento excessivo de determinado assunto ou tópico secundário dentro dos capítulos da dissertação ou da tese. São contribuições que servem para esclarecer, complementar, provar ou confirmar as ideias apresentadas no texto dos capítulos e que são importantes para a compreensão dos mesmos.

Todos os apêndices devem vir após as referências bibliográficas e devem ser enumerados por letras maiúsculas (A, B, C, ...).

%=====================================================

\section{Uma Seção}

\lipsum[20-23]

%=====================================================

\subsection{Uma Subseção}

\lipsum[30-33]

%=====================================================

\subsection{Outra Subseção}

Exemplo de lista simples com dois níveis: Exemplo de lista simples com dois níveis: Exemplo de lista simples com dois níveis: Exemplo de lista simples com dois níveis: Exemplo de lista simples com dois níveis: Exemplo de lista simples com dois níveis: Exemplo de lista simples com dois níveis: Exemplo de lista simples com dois níveis: Exemplo de lista simples com dois níveis.

\begin{itemize}

\item Banana, Banana, Banana, Banana, Banana, Banana, Banana, Banana, Banana, Banana, Banana, Banana, Banana, Banana, Banana, Banana, Banana, Banana, Banana, Banana, Banana, Banana, Banana, Banana.

\begin{itemize}

\item Caturra, Caturra, Caturra, Caturra, Caturra, Caturra, Caturra, Caturra, Caturra, Caturra, Caturra, Caturra, Caturra, Caturra, Caturra, Caturra, Caturra, Caturra, Caturra.

\item da Terra, da Terra, da Terra, da Terra, da Terra, da Terra, da Terra, da Terra, da Terra, da Terra, da Terra, da Terra, da Terra, da Terra, da Terra, da Terra, da Terra, da Terra.

\end{itemize}

\item Laranja, Laranja, Laranja, Laranja, Laranja, Laranja, Laranja, Laranja, Laranja, Laranja, Laranja, Laranja, Laranja, Laranja, Laranja, Laranja.

\begin{itemize}

\item Bahia, Bahia, Bahia, Bahia, Bahia, Bahia, Bahia, Bahia, Bahia, Bahia, Bahia, Bahia, Bahia, Bahia, Bahia, Bahia, Bahia, Bahia.

\item Lima, Lima, Lima, Lima, Lima, Lima, Lima, Lima, Lima, Lima, Lima, Lima, Lima, Lima, Lima, Lima, Lima, Lima, Lima, Lima, Lima, Lima, Lima, Lima, Lima, Lima, Lima, Lima, Lima.

\end{itemize}

\end{itemize}

Exemplo de lista numerada com dois níveis: Exemplo de lista numerada com dois níveis: Exemplo de lista numerada com dois níveis: Exemplo de lista numerada com dois níveis: Exemplo de lista numerada com dois níveis: Exemplo de lista numerada com dois níveis: Exemplo de lista numerada com dois níveis: Exemplo de lista numerada com dois níveis.

\begin{enumerate}

\item Banana, Banana, Banana, Banana, Banana, Banana, Banana, Banana, Banana, Banana, Banana, Banana, Banana, Banana, Banana, Banana, Banana, Banana, Banana, Banana, Banana, Banana, Banana, Banana.

\begin{enumerate}

\item Caturra, Caturra, Caturra, Caturra, Caturra, Caturra, Caturra, Caturra, Caturra, Caturra, Caturra, Caturra, Caturra, Caturra, Caturra, Caturra, Caturra, Caturra, Caturra.

\item da Terra, da Terra, da Terra, da Terra, da Terra, da Terra, da Terra, da Terra, da Terra, da Terra, da Terra, da Terra, da Terra, da Terra, da Terra, da Terra, da Terra, da Terra.

\end{enumerate}

\item Laranja, Laranja, Laranja, Laranja, Laranja, Laranja, Laranja, Laranja, Laranja, Laranja, Laranja, Laranja, Laranja, Laranja, Laranja, Laranja.

\begin{enumerate}

\item Bahia, Bahia, Bahia, Bahia, Bahia, Bahia, Bahia, Bahia, Bahia, Bahia, Bahia, Bahia, Bahia, Bahia, Bahia, Bahia, Bahia, Bahia.

\item Lima, Lima, Lima, Lima, Lima, Lima, Lima, Lima, Lima, Lima, Lima, Lima, Lima, Lima, Lima, Lima, Lima, Lima, Lima, Lima, Lima, Lima, Lima, Lima, Lima, Lima, Lima, Lima, Lima.

\end{enumerate}

\end{enumerate}

Exemplo de lista descritiva com dois níveis: Exemplo de lista descritiva com dois níveis: Exemplo de lista descritiva com dois níveis: Exemplo de lista descritiva com dois níveis: Exemplo de lista descritiva com dois níveis: Exemplo de lista descritiva com dois níveis: Exemplo de lista descritiva com dois níveis: Exemplo de lista descritiva com dois níveis: Exemplo de lista descritiva com dois níveis.

\begin{description}

\item [Banana]: Banana, Banana, Banana, Banana, Banana, Banana, Banana, Banana, Banana, Banana, Banana, Banana, Banana, Banana, Banana, Banana, Banana, Banana, Banana, Banana, Banana, Banana, Banana.

\begin{description}

\item [Caturra]: Caturra, Caturra, Caturra, Caturra, Caturra, Caturra, Caturra, Caturra, Caturra, Caturra, Caturra, Caturra, Caturra, Caturra, Caturra, Caturra, Caturra, Caturra.

\item [da Terra]: da Terra, da Terra, da Terra, da Terra, da Terra, da Terra, da Terra, da Terra, da Terra, da Terra, da Terra, da Terra, da Terra, da Terra, da Terra, da Terra, da Terra.

\end{description}

\item [Laranja]: Laranja, Laranja, Laranja, Laranja, Laranja, Laranja, Laranja, Laranja, Laranja, Laranja, Laranja, Laranja, Laranja, Laranja, Laranja.

\begin{description}

\item [Bahia]: Bahia, Bahia, Bahia, Bahia, Bahia, Bahia, Bahia, Bahia, Bahia, Bahia, Bahia, Bahia, Bahia, Bahia, Bahia, Bahia, Bahia.

\item [Lima]: Lima, Lima, Lima, Lima, Lima, Lima, Lima, Lima, Lima, Lima, Lima, Lima, Lima, Lima, Lima, Lima, Lima, Lima, Lima, Lima, Lima, Lima, Lima, Lima, Lima, Lima, Lima, Lima.

\end{description}

\end{description}

%=====================================================
		% fundamentação teórica
%\chapter{Exemplo de anexo}

%=====================================================

Os apêndices são uma extensão do texto, destacados deste para evitar descontinuidade na sequência lógica ou alongamento excessivo de determinado assunto ou tópico secundário dentro dos capítulos da dissertação ou da tese. São contribuições que servem para esclarecer, complementar, provar ou confirmar as ideias apresentadas no texto dos capítulos e que são importantes para a compreensão dos mesmos.

Todos os apêndices devem vir após as referências bibliográficas e devem ser enumerados por letras maiúsculas (A, B, C, ...).

%=====================================================

\section{Uma Seção}

\lipsum[20-23]

%=====================================================

\subsection{Uma Subseção}

\lipsum[30-33]

%=====================================================

\subsection{Outra Subseção}

Exemplo de lista simples com dois níveis: Exemplo de lista simples com dois níveis: Exemplo de lista simples com dois níveis: Exemplo de lista simples com dois níveis: Exemplo de lista simples com dois níveis: Exemplo de lista simples com dois níveis: Exemplo de lista simples com dois níveis: Exemplo de lista simples com dois níveis: Exemplo de lista simples com dois níveis.

\begin{itemize}

\item Banana, Banana, Banana, Banana, Banana, Banana, Banana, Banana, Banana, Banana, Banana, Banana, Banana, Banana, Banana, Banana, Banana, Banana, Banana, Banana, Banana, Banana, Banana, Banana.

\begin{itemize}

\item Caturra, Caturra, Caturra, Caturra, Caturra, Caturra, Caturra, Caturra, Caturra, Caturra, Caturra, Caturra, Caturra, Caturra, Caturra, Caturra, Caturra, Caturra, Caturra.

\item da Terra, da Terra, da Terra, da Terra, da Terra, da Terra, da Terra, da Terra, da Terra, da Terra, da Terra, da Terra, da Terra, da Terra, da Terra, da Terra, da Terra, da Terra.

\end{itemize}

\item Laranja, Laranja, Laranja, Laranja, Laranja, Laranja, Laranja, Laranja, Laranja, Laranja, Laranja, Laranja, Laranja, Laranja, Laranja, Laranja.

\begin{itemize}

\item Bahia, Bahia, Bahia, Bahia, Bahia, Bahia, Bahia, Bahia, Bahia, Bahia, Bahia, Bahia, Bahia, Bahia, Bahia, Bahia, Bahia, Bahia.

\item Lima, Lima, Lima, Lima, Lima, Lima, Lima, Lima, Lima, Lima, Lima, Lima, Lima, Lima, Lima, Lima, Lima, Lima, Lima, Lima, Lima, Lima, Lima, Lima, Lima, Lima, Lima, Lima, Lima.

\end{itemize}

\end{itemize}

Exemplo de lista numerada com dois níveis: Exemplo de lista numerada com dois níveis: Exemplo de lista numerada com dois níveis: Exemplo de lista numerada com dois níveis: Exemplo de lista numerada com dois níveis: Exemplo de lista numerada com dois níveis: Exemplo de lista numerada com dois níveis: Exemplo de lista numerada com dois níveis.

\begin{enumerate}

\item Banana, Banana, Banana, Banana, Banana, Banana, Banana, Banana, Banana, Banana, Banana, Banana, Banana, Banana, Banana, Banana, Banana, Banana, Banana, Banana, Banana, Banana, Banana, Banana.

\begin{enumerate}

\item Caturra, Caturra, Caturra, Caturra, Caturra, Caturra, Caturra, Caturra, Caturra, Caturra, Caturra, Caturra, Caturra, Caturra, Caturra, Caturra, Caturra, Caturra, Caturra.

\item da Terra, da Terra, da Terra, da Terra, da Terra, da Terra, da Terra, da Terra, da Terra, da Terra, da Terra, da Terra, da Terra, da Terra, da Terra, da Terra, da Terra, da Terra.

\end{enumerate}

\item Laranja, Laranja, Laranja, Laranja, Laranja, Laranja, Laranja, Laranja, Laranja, Laranja, Laranja, Laranja, Laranja, Laranja, Laranja, Laranja.

\begin{enumerate}

\item Bahia, Bahia, Bahia, Bahia, Bahia, Bahia, Bahia, Bahia, Bahia, Bahia, Bahia, Bahia, Bahia, Bahia, Bahia, Bahia, Bahia, Bahia.

\item Lima, Lima, Lima, Lima, Lima, Lima, Lima, Lima, Lima, Lima, Lima, Lima, Lima, Lima, Lima, Lima, Lima, Lima, Lima, Lima, Lima, Lima, Lima, Lima, Lima, Lima, Lima, Lima, Lima.

\end{enumerate}

\end{enumerate}

Exemplo de lista descritiva com dois níveis: Exemplo de lista descritiva com dois níveis: Exemplo de lista descritiva com dois níveis: Exemplo de lista descritiva com dois níveis: Exemplo de lista descritiva com dois níveis: Exemplo de lista descritiva com dois níveis: Exemplo de lista descritiva com dois níveis: Exemplo de lista descritiva com dois níveis: Exemplo de lista descritiva com dois níveis.

\begin{description}

\item [Banana]: Banana, Banana, Banana, Banana, Banana, Banana, Banana, Banana, Banana, Banana, Banana, Banana, Banana, Banana, Banana, Banana, Banana, Banana, Banana, Banana, Banana, Banana, Banana.

\begin{description}

\item [Caturra]: Caturra, Caturra, Caturra, Caturra, Caturra, Caturra, Caturra, Caturra, Caturra, Caturra, Caturra, Caturra, Caturra, Caturra, Caturra, Caturra, Caturra, Caturra.

\item [da Terra]: da Terra, da Terra, da Terra, da Terra, da Terra, da Terra, da Terra, da Terra, da Terra, da Terra, da Terra, da Terra, da Terra, da Terra, da Terra, da Terra, da Terra.

\end{description}

\item [Laranja]: Laranja, Laranja, Laranja, Laranja, Laranja, Laranja, Laranja, Laranja, Laranja, Laranja, Laranja, Laranja, Laranja, Laranja, Laranja.

\begin{description}

\item [Bahia]: Bahia, Bahia, Bahia, Bahia, Bahia, Bahia, Bahia, Bahia, Bahia, Bahia, Bahia, Bahia, Bahia, Bahia, Bahia, Bahia, Bahia.

\item [Lima]: Lima, Lima, Lima, Lima, Lima, Lima, Lima, Lima, Lima, Lima, Lima, Lima, Lima, Lima, Lima, Lima, Lima, Lima, Lima, Lima, Lima, Lima, Lima, Lima, Lima, Lima, Lima, Lima.

\end{description}

\end{description}

%=====================================================
		% revisão bibliográfica (estado da arte)
%\chapter{Exemplo de anexo}

%=====================================================

Os apêndices são uma extensão do texto, destacados deste para evitar descontinuidade na sequência lógica ou alongamento excessivo de determinado assunto ou tópico secundário dentro dos capítulos da dissertação ou da tese. São contribuições que servem para esclarecer, complementar, provar ou confirmar as ideias apresentadas no texto dos capítulos e que são importantes para a compreensão dos mesmos.

Todos os apêndices devem vir após as referências bibliográficas e devem ser enumerados por letras maiúsculas (A, B, C, ...).

%=====================================================

\section{Uma Seção}

\lipsum[20-23]

%=====================================================

\subsection{Uma Subseção}

\lipsum[30-33]

%=====================================================

\subsection{Outra Subseção}

Exemplo de lista simples com dois níveis: Exemplo de lista simples com dois níveis: Exemplo de lista simples com dois níveis: Exemplo de lista simples com dois níveis: Exemplo de lista simples com dois níveis: Exemplo de lista simples com dois níveis: Exemplo de lista simples com dois níveis: Exemplo de lista simples com dois níveis: Exemplo de lista simples com dois níveis.

\begin{itemize}

\item Banana, Banana, Banana, Banana, Banana, Banana, Banana, Banana, Banana, Banana, Banana, Banana, Banana, Banana, Banana, Banana, Banana, Banana, Banana, Banana, Banana, Banana, Banana, Banana.

\begin{itemize}

\item Caturra, Caturra, Caturra, Caturra, Caturra, Caturra, Caturra, Caturra, Caturra, Caturra, Caturra, Caturra, Caturra, Caturra, Caturra, Caturra, Caturra, Caturra, Caturra.

\item da Terra, da Terra, da Terra, da Terra, da Terra, da Terra, da Terra, da Terra, da Terra, da Terra, da Terra, da Terra, da Terra, da Terra, da Terra, da Terra, da Terra, da Terra.

\end{itemize}

\item Laranja, Laranja, Laranja, Laranja, Laranja, Laranja, Laranja, Laranja, Laranja, Laranja, Laranja, Laranja, Laranja, Laranja, Laranja, Laranja.

\begin{itemize}

\item Bahia, Bahia, Bahia, Bahia, Bahia, Bahia, Bahia, Bahia, Bahia, Bahia, Bahia, Bahia, Bahia, Bahia, Bahia, Bahia, Bahia, Bahia.

\item Lima, Lima, Lima, Lima, Lima, Lima, Lima, Lima, Lima, Lima, Lima, Lima, Lima, Lima, Lima, Lima, Lima, Lima, Lima, Lima, Lima, Lima, Lima, Lima, Lima, Lima, Lima, Lima, Lima.

\end{itemize}

\end{itemize}

Exemplo de lista numerada com dois níveis: Exemplo de lista numerada com dois níveis: Exemplo de lista numerada com dois níveis: Exemplo de lista numerada com dois níveis: Exemplo de lista numerada com dois níveis: Exemplo de lista numerada com dois níveis: Exemplo de lista numerada com dois níveis: Exemplo de lista numerada com dois níveis.

\begin{enumerate}

\item Banana, Banana, Banana, Banana, Banana, Banana, Banana, Banana, Banana, Banana, Banana, Banana, Banana, Banana, Banana, Banana, Banana, Banana, Banana, Banana, Banana, Banana, Banana, Banana.

\begin{enumerate}

\item Caturra, Caturra, Caturra, Caturra, Caturra, Caturra, Caturra, Caturra, Caturra, Caturra, Caturra, Caturra, Caturra, Caturra, Caturra, Caturra, Caturra, Caturra, Caturra.

\item da Terra, da Terra, da Terra, da Terra, da Terra, da Terra, da Terra, da Terra, da Terra, da Terra, da Terra, da Terra, da Terra, da Terra, da Terra, da Terra, da Terra, da Terra.

\end{enumerate}

\item Laranja, Laranja, Laranja, Laranja, Laranja, Laranja, Laranja, Laranja, Laranja, Laranja, Laranja, Laranja, Laranja, Laranja, Laranja, Laranja.

\begin{enumerate}

\item Bahia, Bahia, Bahia, Bahia, Bahia, Bahia, Bahia, Bahia, Bahia, Bahia, Bahia, Bahia, Bahia, Bahia, Bahia, Bahia, Bahia, Bahia.

\item Lima, Lima, Lima, Lima, Lima, Lima, Lima, Lima, Lima, Lima, Lima, Lima, Lima, Lima, Lima, Lima, Lima, Lima, Lima, Lima, Lima, Lima, Lima, Lima, Lima, Lima, Lima, Lima, Lima.

\end{enumerate}

\end{enumerate}

Exemplo de lista descritiva com dois níveis: Exemplo de lista descritiva com dois níveis: Exemplo de lista descritiva com dois níveis: Exemplo de lista descritiva com dois níveis: Exemplo de lista descritiva com dois níveis: Exemplo de lista descritiva com dois níveis: Exemplo de lista descritiva com dois níveis: Exemplo de lista descritiva com dois níveis: Exemplo de lista descritiva com dois níveis.

\begin{description}

\item [Banana]: Banana, Banana, Banana, Banana, Banana, Banana, Banana, Banana, Banana, Banana, Banana, Banana, Banana, Banana, Banana, Banana, Banana, Banana, Banana, Banana, Banana, Banana, Banana.

\begin{description}

\item [Caturra]: Caturra, Caturra, Caturra, Caturra, Caturra, Caturra, Caturra, Caturra, Caturra, Caturra, Caturra, Caturra, Caturra, Caturra, Caturra, Caturra, Caturra, Caturra.

\item [da Terra]: da Terra, da Terra, da Terra, da Terra, da Terra, da Terra, da Terra, da Terra, da Terra, da Terra, da Terra, da Terra, da Terra, da Terra, da Terra, da Terra, da Terra.

\end{description}

\item [Laranja]: Laranja, Laranja, Laranja, Laranja, Laranja, Laranja, Laranja, Laranja, Laranja, Laranja, Laranja, Laranja, Laranja, Laranja, Laranja.

\begin{description}

\item [Bahia]: Bahia, Bahia, Bahia, Bahia, Bahia, Bahia, Bahia, Bahia, Bahia, Bahia, Bahia, Bahia, Bahia, Bahia, Bahia, Bahia, Bahia.

\item [Lima]: Lima, Lima, Lima, Lima, Lima, Lima, Lima, Lima, Lima, Lima, Lima, Lima, Lima, Lima, Lima, Lima, Lima, Lima, Lima, Lima, Lima, Lima, Lima, Lima, Lima, Lima, Lima, Lima.

\end{description}

\end{description}

%=====================================================
		% proposta
%\chapter{Exemplo de anexo}

%=====================================================

Os apêndices são uma extensão do texto, destacados deste para evitar descontinuidade na sequência lógica ou alongamento excessivo de determinado assunto ou tópico secundário dentro dos capítulos da dissertação ou da tese. São contribuições que servem para esclarecer, complementar, provar ou confirmar as ideias apresentadas no texto dos capítulos e que são importantes para a compreensão dos mesmos.

Todos os apêndices devem vir após as referências bibliográficas e devem ser enumerados por letras maiúsculas (A, B, C, ...).

%=====================================================

\section{Uma Seção}

\lipsum[20-23]

%=====================================================

\subsection{Uma Subseção}

\lipsum[30-33]

%=====================================================

\subsection{Outra Subseção}

Exemplo de lista simples com dois níveis: Exemplo de lista simples com dois níveis: Exemplo de lista simples com dois níveis: Exemplo de lista simples com dois níveis: Exemplo de lista simples com dois níveis: Exemplo de lista simples com dois níveis: Exemplo de lista simples com dois níveis: Exemplo de lista simples com dois níveis: Exemplo de lista simples com dois níveis.

\begin{itemize}

\item Banana, Banana, Banana, Banana, Banana, Banana, Banana, Banana, Banana, Banana, Banana, Banana, Banana, Banana, Banana, Banana, Banana, Banana, Banana, Banana, Banana, Banana, Banana, Banana.

\begin{itemize}

\item Caturra, Caturra, Caturra, Caturra, Caturra, Caturra, Caturra, Caturra, Caturra, Caturra, Caturra, Caturra, Caturra, Caturra, Caturra, Caturra, Caturra, Caturra, Caturra.

\item da Terra, da Terra, da Terra, da Terra, da Terra, da Terra, da Terra, da Terra, da Terra, da Terra, da Terra, da Terra, da Terra, da Terra, da Terra, da Terra, da Terra, da Terra.

\end{itemize}

\item Laranja, Laranja, Laranja, Laranja, Laranja, Laranja, Laranja, Laranja, Laranja, Laranja, Laranja, Laranja, Laranja, Laranja, Laranja, Laranja.

\begin{itemize}

\item Bahia, Bahia, Bahia, Bahia, Bahia, Bahia, Bahia, Bahia, Bahia, Bahia, Bahia, Bahia, Bahia, Bahia, Bahia, Bahia, Bahia, Bahia.

\item Lima, Lima, Lima, Lima, Lima, Lima, Lima, Lima, Lima, Lima, Lima, Lima, Lima, Lima, Lima, Lima, Lima, Lima, Lima, Lima, Lima, Lima, Lima, Lima, Lima, Lima, Lima, Lima, Lima.

\end{itemize}

\end{itemize}

Exemplo de lista numerada com dois níveis: Exemplo de lista numerada com dois níveis: Exemplo de lista numerada com dois níveis: Exemplo de lista numerada com dois níveis: Exemplo de lista numerada com dois níveis: Exemplo de lista numerada com dois níveis: Exemplo de lista numerada com dois níveis: Exemplo de lista numerada com dois níveis.

\begin{enumerate}

\item Banana, Banana, Banana, Banana, Banana, Banana, Banana, Banana, Banana, Banana, Banana, Banana, Banana, Banana, Banana, Banana, Banana, Banana, Banana, Banana, Banana, Banana, Banana, Banana.

\begin{enumerate}

\item Caturra, Caturra, Caturra, Caturra, Caturra, Caturra, Caturra, Caturra, Caturra, Caturra, Caturra, Caturra, Caturra, Caturra, Caturra, Caturra, Caturra, Caturra, Caturra.

\item da Terra, da Terra, da Terra, da Terra, da Terra, da Terra, da Terra, da Terra, da Terra, da Terra, da Terra, da Terra, da Terra, da Terra, da Terra, da Terra, da Terra, da Terra.

\end{enumerate}

\item Laranja, Laranja, Laranja, Laranja, Laranja, Laranja, Laranja, Laranja, Laranja, Laranja, Laranja, Laranja, Laranja, Laranja, Laranja, Laranja.

\begin{enumerate}

\item Bahia, Bahia, Bahia, Bahia, Bahia, Bahia, Bahia, Bahia, Bahia, Bahia, Bahia, Bahia, Bahia, Bahia, Bahia, Bahia, Bahia, Bahia.

\item Lima, Lima, Lima, Lima, Lima, Lima, Lima, Lima, Lima, Lima, Lima, Lima, Lima, Lima, Lima, Lima, Lima, Lima, Lima, Lima, Lima, Lima, Lima, Lima, Lima, Lima, Lima, Lima, Lima.

\end{enumerate}

\end{enumerate}

Exemplo de lista descritiva com dois níveis: Exemplo de lista descritiva com dois níveis: Exemplo de lista descritiva com dois níveis: Exemplo de lista descritiva com dois níveis: Exemplo de lista descritiva com dois níveis: Exemplo de lista descritiva com dois níveis: Exemplo de lista descritiva com dois níveis: Exemplo de lista descritiva com dois níveis: Exemplo de lista descritiva com dois níveis.

\begin{description}

\item [Banana]: Banana, Banana, Banana, Banana, Banana, Banana, Banana, Banana, Banana, Banana, Banana, Banana, Banana, Banana, Banana, Banana, Banana, Banana, Banana, Banana, Banana, Banana, Banana.

\begin{description}

\item [Caturra]: Caturra, Caturra, Caturra, Caturra, Caturra, Caturra, Caturra, Caturra, Caturra, Caturra, Caturra, Caturra, Caturra, Caturra, Caturra, Caturra, Caturra, Caturra.

\item [da Terra]: da Terra, da Terra, da Terra, da Terra, da Terra, da Terra, da Terra, da Terra, da Terra, da Terra, da Terra, da Terra, da Terra, da Terra, da Terra, da Terra, da Terra.

\end{description}

\item [Laranja]: Laranja, Laranja, Laranja, Laranja, Laranja, Laranja, Laranja, Laranja, Laranja, Laranja, Laranja, Laranja, Laranja, Laranja, Laranja.

\begin{description}

\item [Bahia]: Bahia, Bahia, Bahia, Bahia, Bahia, Bahia, Bahia, Bahia, Bahia, Bahia, Bahia, Bahia, Bahia, Bahia, Bahia, Bahia, Bahia.

\item [Lima]: Lima, Lima, Lima, Lima, Lima, Lima, Lima, Lima, Lima, Lima, Lima, Lima, Lima, Lima, Lima, Lima, Lima, Lima, Lima, Lima, Lima, Lima, Lima, Lima, Lima, Lima, Lima, Lima.

\end{description}

\end{description}

%=====================================================
		% experimentação e validação
%\chapter{Exemplo de anexo}

%=====================================================

Os apêndices são uma extensão do texto, destacados deste para evitar descontinuidade na sequência lógica ou alongamento excessivo de determinado assunto ou tópico secundário dentro dos capítulos da dissertação ou da tese. São contribuições que servem para esclarecer, complementar, provar ou confirmar as ideias apresentadas no texto dos capítulos e que são importantes para a compreensão dos mesmos.

Todos os apêndices devem vir após as referências bibliográficas e devem ser enumerados por letras maiúsculas (A, B, C, ...).

%=====================================================

\section{Uma Seção}

\lipsum[20-23]

%=====================================================

\subsection{Uma Subseção}

\lipsum[30-33]

%=====================================================

\subsection{Outra Subseção}

Exemplo de lista simples com dois níveis: Exemplo de lista simples com dois níveis: Exemplo de lista simples com dois níveis: Exemplo de lista simples com dois níveis: Exemplo de lista simples com dois níveis: Exemplo de lista simples com dois níveis: Exemplo de lista simples com dois níveis: Exemplo de lista simples com dois níveis: Exemplo de lista simples com dois níveis.

\begin{itemize}

\item Banana, Banana, Banana, Banana, Banana, Banana, Banana, Banana, Banana, Banana, Banana, Banana, Banana, Banana, Banana, Banana, Banana, Banana, Banana, Banana, Banana, Banana, Banana, Banana.

\begin{itemize}

\item Caturra, Caturra, Caturra, Caturra, Caturra, Caturra, Caturra, Caturra, Caturra, Caturra, Caturra, Caturra, Caturra, Caturra, Caturra, Caturra, Caturra, Caturra, Caturra.

\item da Terra, da Terra, da Terra, da Terra, da Terra, da Terra, da Terra, da Terra, da Terra, da Terra, da Terra, da Terra, da Terra, da Terra, da Terra, da Terra, da Terra, da Terra.

\end{itemize}

\item Laranja, Laranja, Laranja, Laranja, Laranja, Laranja, Laranja, Laranja, Laranja, Laranja, Laranja, Laranja, Laranja, Laranja, Laranja, Laranja.

\begin{itemize}

\item Bahia, Bahia, Bahia, Bahia, Bahia, Bahia, Bahia, Bahia, Bahia, Bahia, Bahia, Bahia, Bahia, Bahia, Bahia, Bahia, Bahia, Bahia.

\item Lima, Lima, Lima, Lima, Lima, Lima, Lima, Lima, Lima, Lima, Lima, Lima, Lima, Lima, Lima, Lima, Lima, Lima, Lima, Lima, Lima, Lima, Lima, Lima, Lima, Lima, Lima, Lima, Lima.

\end{itemize}

\end{itemize}

Exemplo de lista numerada com dois níveis: Exemplo de lista numerada com dois níveis: Exemplo de lista numerada com dois níveis: Exemplo de lista numerada com dois níveis: Exemplo de lista numerada com dois níveis: Exemplo de lista numerada com dois níveis: Exemplo de lista numerada com dois níveis: Exemplo de lista numerada com dois níveis.

\begin{enumerate}

\item Banana, Banana, Banana, Banana, Banana, Banana, Banana, Banana, Banana, Banana, Banana, Banana, Banana, Banana, Banana, Banana, Banana, Banana, Banana, Banana, Banana, Banana, Banana, Banana.

\begin{enumerate}

\item Caturra, Caturra, Caturra, Caturra, Caturra, Caturra, Caturra, Caturra, Caturra, Caturra, Caturra, Caturra, Caturra, Caturra, Caturra, Caturra, Caturra, Caturra, Caturra.

\item da Terra, da Terra, da Terra, da Terra, da Terra, da Terra, da Terra, da Terra, da Terra, da Terra, da Terra, da Terra, da Terra, da Terra, da Terra, da Terra, da Terra, da Terra.

\end{enumerate}

\item Laranja, Laranja, Laranja, Laranja, Laranja, Laranja, Laranja, Laranja, Laranja, Laranja, Laranja, Laranja, Laranja, Laranja, Laranja, Laranja.

\begin{enumerate}

\item Bahia, Bahia, Bahia, Bahia, Bahia, Bahia, Bahia, Bahia, Bahia, Bahia, Bahia, Bahia, Bahia, Bahia, Bahia, Bahia, Bahia, Bahia.

\item Lima, Lima, Lima, Lima, Lima, Lima, Lima, Lima, Lima, Lima, Lima, Lima, Lima, Lima, Lima, Lima, Lima, Lima, Lima, Lima, Lima, Lima, Lima, Lima, Lima, Lima, Lima, Lima, Lima.

\end{enumerate}

\end{enumerate}

Exemplo de lista descritiva com dois níveis: Exemplo de lista descritiva com dois níveis: Exemplo de lista descritiva com dois níveis: Exemplo de lista descritiva com dois níveis: Exemplo de lista descritiva com dois níveis: Exemplo de lista descritiva com dois níveis: Exemplo de lista descritiva com dois níveis: Exemplo de lista descritiva com dois níveis: Exemplo de lista descritiva com dois níveis.

\begin{description}

\item [Banana]: Banana, Banana, Banana, Banana, Banana, Banana, Banana, Banana, Banana, Banana, Banana, Banana, Banana, Banana, Banana, Banana, Banana, Banana, Banana, Banana, Banana, Banana, Banana.

\begin{description}

\item [Caturra]: Caturra, Caturra, Caturra, Caturra, Caturra, Caturra, Caturra, Caturra, Caturra, Caturra, Caturra, Caturra, Caturra, Caturra, Caturra, Caturra, Caturra, Caturra.

\item [da Terra]: da Terra, da Terra, da Terra, da Terra, da Terra, da Terra, da Terra, da Terra, da Terra, da Terra, da Terra, da Terra, da Terra, da Terra, da Terra, da Terra, da Terra.

\end{description}

\item [Laranja]: Laranja, Laranja, Laranja, Laranja, Laranja, Laranja, Laranja, Laranja, Laranja, Laranja, Laranja, Laranja, Laranja, Laranja, Laranja.

\begin{description}

\item [Bahia]: Bahia, Bahia, Bahia, Bahia, Bahia, Bahia, Bahia, Bahia, Bahia, Bahia, Bahia, Bahia, Bahia, Bahia, Bahia, Bahia, Bahia.

\item [Lima]: Lima, Lima, Lima, Lima, Lima, Lima, Lima, Lima, Lima, Lima, Lima, Lima, Lima, Lima, Lima, Lima, Lima, Lima, Lima, Lima, Lima, Lima, Lima, Lima, Lima, Lima, Lima, Lima.

\end{description}

\end{description}

%=====================================================
		% conclusão

%=====================================================

% Estilos de bibliografia recomendados (só descomentar um estilo!)
% Mais infos: https://pt.sharelatex.com/learn/Bibtex_bibliography_styles
\bibliographystyle{apalike-ptbr}	% [Maziero et al., 2006]
%\bibliographystyle{alpha}		% [Maz06]
%\bibliographystyle{plainnat}		% vide Google "LaTeX Natbib"
%\bibliographystyle{plain}		% [1] ordem alfabética
%\bibliographystyle{unsrt}		% [1] ordem de uso no texto

% no estilo "unsrt", evita que citações nos índices sejam consideradas
%\usepackage{notoccite}

% base de bibliografia (BibTeX)
\bibliography{referencias}
%\bibliography{file1, file2, file3} % se tiver mais de um arquivo BibTeX

%=====================================================

% inclusão de apêndices
\appendix
\chapter{Exemplo de anexo}

%=====================================================

Os apêndices são uma extensão do texto, destacados deste para evitar descontinuidade na sequência lógica ou alongamento excessivo de determinado assunto ou tópico secundário dentro dos capítulos da dissertação ou da tese. São contribuições que servem para esclarecer, complementar, provar ou confirmar as ideias apresentadas no texto dos capítulos e que são importantes para a compreensão dos mesmos.

Todos os apêndices devem vir após as referências bibliográficas e devem ser enumerados por letras maiúsculas (A, B, C, ...).

%=====================================================

\section{Uma Seção}

\lipsum[20-23]

%=====================================================

\subsection{Uma Subseção}

\lipsum[30-33]

%=====================================================

\subsection{Outra Subseção}

Exemplo de lista simples com dois níveis: Exemplo de lista simples com dois níveis: Exemplo de lista simples com dois níveis: Exemplo de lista simples com dois níveis: Exemplo de lista simples com dois níveis: Exemplo de lista simples com dois níveis: Exemplo de lista simples com dois níveis: Exemplo de lista simples com dois níveis: Exemplo de lista simples com dois níveis.

\begin{itemize}

\item Banana, Banana, Banana, Banana, Banana, Banana, Banana, Banana, Banana, Banana, Banana, Banana, Banana, Banana, Banana, Banana, Banana, Banana, Banana, Banana, Banana, Banana, Banana, Banana.

\begin{itemize}

\item Caturra, Caturra, Caturra, Caturra, Caturra, Caturra, Caturra, Caturra, Caturra, Caturra, Caturra, Caturra, Caturra, Caturra, Caturra, Caturra, Caturra, Caturra, Caturra.

\item da Terra, da Terra, da Terra, da Terra, da Terra, da Terra, da Terra, da Terra, da Terra, da Terra, da Terra, da Terra, da Terra, da Terra, da Terra, da Terra, da Terra, da Terra.

\end{itemize}

\item Laranja, Laranja, Laranja, Laranja, Laranja, Laranja, Laranja, Laranja, Laranja, Laranja, Laranja, Laranja, Laranja, Laranja, Laranja, Laranja.

\begin{itemize}

\item Bahia, Bahia, Bahia, Bahia, Bahia, Bahia, Bahia, Bahia, Bahia, Bahia, Bahia, Bahia, Bahia, Bahia, Bahia, Bahia, Bahia, Bahia.

\item Lima, Lima, Lima, Lima, Lima, Lima, Lima, Lima, Lima, Lima, Lima, Lima, Lima, Lima, Lima, Lima, Lima, Lima, Lima, Lima, Lima, Lima, Lima, Lima, Lima, Lima, Lima, Lima, Lima.

\end{itemize}

\end{itemize}

Exemplo de lista numerada com dois níveis: Exemplo de lista numerada com dois níveis: Exemplo de lista numerada com dois níveis: Exemplo de lista numerada com dois níveis: Exemplo de lista numerada com dois níveis: Exemplo de lista numerada com dois níveis: Exemplo de lista numerada com dois níveis: Exemplo de lista numerada com dois níveis.

\begin{enumerate}

\item Banana, Banana, Banana, Banana, Banana, Banana, Banana, Banana, Banana, Banana, Banana, Banana, Banana, Banana, Banana, Banana, Banana, Banana, Banana, Banana, Banana, Banana, Banana, Banana.

\begin{enumerate}

\item Caturra, Caturra, Caturra, Caturra, Caturra, Caturra, Caturra, Caturra, Caturra, Caturra, Caturra, Caturra, Caturra, Caturra, Caturra, Caturra, Caturra, Caturra, Caturra.

\item da Terra, da Terra, da Terra, da Terra, da Terra, da Terra, da Terra, da Terra, da Terra, da Terra, da Terra, da Terra, da Terra, da Terra, da Terra, da Terra, da Terra, da Terra.

\end{enumerate}

\item Laranja, Laranja, Laranja, Laranja, Laranja, Laranja, Laranja, Laranja, Laranja, Laranja, Laranja, Laranja, Laranja, Laranja, Laranja, Laranja.

\begin{enumerate}

\item Bahia, Bahia, Bahia, Bahia, Bahia, Bahia, Bahia, Bahia, Bahia, Bahia, Bahia, Bahia, Bahia, Bahia, Bahia, Bahia, Bahia, Bahia.

\item Lima, Lima, Lima, Lima, Lima, Lima, Lima, Lima, Lima, Lima, Lima, Lima, Lima, Lima, Lima, Lima, Lima, Lima, Lima, Lima, Lima, Lima, Lima, Lima, Lima, Lima, Lima, Lima, Lima.

\end{enumerate}

\end{enumerate}

Exemplo de lista descritiva com dois níveis: Exemplo de lista descritiva com dois níveis: Exemplo de lista descritiva com dois níveis: Exemplo de lista descritiva com dois níveis: Exemplo de lista descritiva com dois níveis: Exemplo de lista descritiva com dois níveis: Exemplo de lista descritiva com dois níveis: Exemplo de lista descritiva com dois níveis: Exemplo de lista descritiva com dois níveis.

\begin{description}

\item [Banana]: Banana, Banana, Banana, Banana, Banana, Banana, Banana, Banana, Banana, Banana, Banana, Banana, Banana, Banana, Banana, Banana, Banana, Banana, Banana, Banana, Banana, Banana, Banana.

\begin{description}

\item [Caturra]: Caturra, Caturra, Caturra, Caturra, Caturra, Caturra, Caturra, Caturra, Caturra, Caturra, Caturra, Caturra, Caturra, Caturra, Caturra, Caturra, Caturra, Caturra.

\item [da Terra]: da Terra, da Terra, da Terra, da Terra, da Terra, da Terra, da Terra, da Terra, da Terra, da Terra, da Terra, da Terra, da Terra, da Terra, da Terra, da Terra, da Terra.

\end{description}

\item [Laranja]: Laranja, Laranja, Laranja, Laranja, Laranja, Laranja, Laranja, Laranja, Laranja, Laranja, Laranja, Laranja, Laranja, Laranja, Laranja.

\begin{description}

\item [Bahia]: Bahia, Bahia, Bahia, Bahia, Bahia, Bahia, Bahia, Bahia, Bahia, Bahia, Bahia, Bahia, Bahia, Bahia, Bahia, Bahia, Bahia.

\item [Lima]: Lima, Lima, Lima, Lima, Lima, Lima, Lima, Lima, Lima, Lima, Lima, Lima, Lima, Lima, Lima, Lima, Lima, Lima, Lima, Lima, Lima, Lima, Lima, Lima, Lima, Lima, Lima, Lima.

\end{description}

\end{description}

%=====================================================


\end{document}

%=====================================================
